\chapter{ARCHIVE}
The \lstinline;ARCHIVE; class is the central piece of this library. It allows
clients to manipulate archives.

\section{Initialization}
To create a new instance of the \lstinline;ARCHIVE; class, the client has to use
the make feature and provide a \lstinline;STORAGE_BACKEND; which will then be
used to do I/O. A client then can either open the archive for archiving or for
unarchiving. It is not possible to archive and unarchive simultaneously.

\section{Archiving}
To use the archiving mode, one has to use the \lstinline;open_archive; feature
of \lstinline;ARCHIVE;. Then, one can add entries using \lstinline;add_entry;.
Once all entries are written, the user has to call finalize, which will write
the end-of-archive indicator and then close the archive.

\section{Unarchiving}
To use the unarchiving mode, one has to call \lstinline;open_unarchive; after
creation. Next, the client may register arbitrarily many \lstinline;UNARCHIVER;s
using \lstinline;add_unarchiver;. Once all of them were added, the client may
either use \lstinline;unarchive; which will then unarchive all entries and close
the archive or \lstinline;unarchive_next_entry; until
\lstinline;unarchiving_finished; becomes \lstinline;True;. In the latter case,
the client has to call \lstinline;close; himself.

